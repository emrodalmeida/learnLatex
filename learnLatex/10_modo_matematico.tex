% https://www.learnlatex.org/pt/lesson-10
% https://www.tug.org/TUGboat/tb41-1/tb127gregorio-math.pdf
% http://texdoc.org/pkg/amsmath

\documentclass{article}
\usepackage[T1]{fontenc}
% \usepackage{amsmath}
\usepackage{mathtools}   % carrega o amsmath e acrescenta funcionalidades adicionais

\begin{document}
Uma frase com matemática embutida: $y = mx + c$.
Uma segunda frase com matemática embutida: $5^{2}=3^{2}+4^{2}$.

Um segundo parágrafo com matemática em destaque e sem numeração:
\[
y = mx + c
\]
Veja como o parágrafo continua após a equação.

Um parágrafo sobre uma equação maior não numerada.
\[
\int_{-\infty}^{+\infty} e^{-x^2} \, dx
\]

Um parágrafo sobre uma equação maior numerada.
\begin{equation}
	\int_{-\infty}^{+\infty} e^{-x^2} \, dx
\end{equation}



\newcommand{\diff}{\mathop{} \! d}            % Itálico
%\newcommand{\diff}{\mathop{} \! \mathrm{d}}  % Normal
Um parágrafo sobre uma equação mais longa
\[
\int_{-\infty}^{+\infty} e^{-x^2} \diff x
\]


Matrizes AMS.
\[
\begin{matrix}
	a & b & c \\
	d & e & f
\end{matrix}
\qquad
\begin{pmatrix}
	a & b & c \\
	d & e & f
\end{pmatrix}
\qquad
\begin{bmatrix}
	a & b & c \\
	d & e & f
\end{bmatrix}
\]


A matriz $\mathbf{M}$.

$\text{bad use } size  \neq \mathit{size} \neq \mathrm{size} $

\textit{$\text{bad use } size \neq \mathit{size} \neq \mathrm{size} $}


% ===================================================

\section{Ambientes de alinhamento do amsmath}


\textbf{gather} para equações de várias linhas que não precisam alinhamento

\begin{gather}
	P(x) = ax^{5} + bx^{4} + cx^{3} + dx^{2} + ex + f \\
	x^2 + x = 10
\end{gather}

\textbf{multiline} para dividir uma única expressão em várias linhas, alinhando a primeira linha à esquerda e a última à direita.

\begin{multline}
	(a + b + c + d) x^{5} + (b + c + d + e) x^{4} \\
	+ (c + d + e + f) x^{3} + (d + e + f + a) x^{2} + (e + f + a + b) x\\
	+ (f + a + b + c)
\end{multline}


% ===================================================

\section{Alinhamento em colunas}

\begin{align}
	a &= b+1   &  c &= d+2  &  e &= f+3   \\
	r &= s^{2} &  t &=u^{3} &  v &= w^{4}
\end{align}

Há variantes dos ambientes com nome terminando em ed, que formam um sub termo dentro de uma equação maior. Por exemplo, \textbf{aligned} e \textbf{gathered} são variantes de \textbf{align} e \textbf{gather} respectivamente.

O ambiente \textbf{aligned} é usado dentro de outro ambiente matemático maior, \textbf{\textbackslash equation} ou estruturas delimitadoras como \textbf{\textbackslash left} e \textbf{\textbackslash right}. Ele não cria sua própria estrutura de exibição, mas organiza o conteúdo dentro do ambiente que o contém.

\[     % ambiente matemático maior
	\left.     			   % abre o bloco com um delimitador invisível
		\begin{aligned}    % organiza o conteúdo dentro do ambiente
		a &= b\\
		c &= d
		\end{aligned} 
	\right\}    		   % fecha o bloco com uma chave direita
	\Longrightarrow
	\left\{      		   % abre o bloco com uma chave esquerda
		\begin{aligned}    % organiza o conteúdo dentro do ambiente
			b &= a\\
			d &= c
		\end{aligned}
	\right.	   			   % fecha o bloco com um delimitador invisível
\]

\textbf{aligned} aceita um argumento opcional de posicionamento similar ao ambiente \textbf{tabular}. Isso geralmente é útil para alinhar uma equação embutida na primeira linha

\begin{itemize}
	\item 
	$\begin{aligned}[t]
		a&=b\\
		c&=d
	\end{aligned}$
	\item 
	$\begin{aligned}
		a&=b\\
		c&=d
	\end{aligned}$
\end{itemize}

% ===================================================

\section{Mathtools}

Permite escolher o alinhamento das colunas da matriz.

\[
\begin{pmatrix*}[r]
	10&11\\
	1&2\\
	-5&-6
\end{pmatrix*}
\]

\end{document}















