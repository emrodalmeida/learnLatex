% https://www.learnlatex.org/pt/more-07

\documentclass[brazilian]{article}
\usepackage[T1]{fontenc}
\usepackage{graphicx}
\usepackage{babel}

\graphicspath{{figs/}{pics/}} % um item entre chaves para cada subpasta
\usepackage{float}  % coloca a figura exatamente onde ela está no código fonte


\usepackage{trivfloat}
\trivfloat{image}  % fornece um único comando para fazer novos tipos de floats.

\begin{document}
A opção \textbf{H} coloca a figura ‘absolutamente aqui’. No entanto não é recomendado usar \textbf{H} pois ele pode criar grandes espaços em branco no seu documento.

\begin{figure}[H]
  \centering
  \includegraphics[width=0.5\textwidth]{example-image}
  \caption{Uma imagem de exemplo}
\end{figure}

Não temos que necessariamente colocar figuras no ambiente \textit{figure} nem tabelas no ambiente \textit{table}; isto é apenas convenção.

Você pode querer outros tipos de ambientes flutuantes (\textbf{floats}); cada tipo é inserido independentemente. Você pode fazer isso usando o pacote \textbf{trivfloat}.
\begin{image}
  \centering
  \includegraphics[width=0.5\textwidth]{example-image}
  \caption{Uma imagem de exemplo}
\end{image}

\end{document}












