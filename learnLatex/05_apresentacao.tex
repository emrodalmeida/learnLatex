% https://www.learnlatex.org/pt/more-05
% https://www.texdev.net/2014/01/17/the-beamer-slide-overlay-concept/


\documentclass{beamer}
\usepackage[T1]{fontenc}

\begin{document}

\begin{frame}
    \frametitle{1º quadro}
    Algum texto
\end{frame}

\begin{frame}
    \frametitle{2º quadro}
    Mais texto
    \begin{itemize}
        \item<1-> Este item aparece no primeiro e todos os slides seguintes.
        \item<2-> Este item aparece no segundo e todos os slides seguintes.
        \item<3-> Este item aparece no terceiro e todos os slides seguintes.        
    \end{itemize}
\end{frame}

\begin{frame}
    \frametitle{3º quadro - Overlay básico}
    \begin{itemize}
        \item<1> Este item aparece só no primeiro slide.
        \item<-3> Este item aparece só nos três primeiros slides.
        \item<2-4,6> Este item aparece no segundo ao quarto slides e depois no sexto slide.

        % The syntax is quite powerful, but there are at least a couple of issues. First, the slide numbers are hard-coded. That means that if I want to add something else in before the first item I’ve got to renumber everything. Secondly, I’m having to repeat myself. Luckily, beamer offers a way to address both of these concerns.
        
    \end{itemize}
\end{frame}

\begin{frame}
    \frametitle{4º Quadro: Auto-incremento}
    \begin{itemize}
        \item<+-> Este item aparece no primeiro e todos os slides seguintes.
        \item<+-> Este item aparece no segundo e todos os slides seguintes.
        \item<+-> Este item aparece no terceiro e todos os slides seguintes.
    \end{itemize}
    
\end{frame}

\begin{frame}
    \frametitle{5º quadro: Reduzindo a redundância}
    \begin{itemize}[<+->]
        \item Este item aparece no primeiro e todos os slides seguintes.
        \item Este item aparece no segundo e todos os slides seguintes.
        \item Este item aparece no terceiro e todos os slides seguintes.
    \end{itemize}
\end{frame}

\begin{frame}
    \frametitle{6º quadro: Reduzindo a redundância}
    \begin{itemize}[<+->]
        \item Este item aparece no primeiro e todos os slides seguintes.
        \item Este item aparece no segundo e todos os slides seguintes.
        \item Este item aparece no terceiro e todos os slides seguintes.
        \item<1-> Este item aparece no primeiro e todos os slides seguintes.
    \end{itemize}
\end{frame}

\begin{frame}{7º quadro: Repetindo a numeração do overlay (manual)}
    \begin{itemize}
        \item<1-> Este item aparece no primeiro e todos os slides seguintes.
        \item<1-> Este item também aparece no primeiro e todos os slides seguintes.
        \item<2-> Este item aparece no segundo e todos os slides seguintes.
        \item<2-> Este item também aparece no segundo e todos os slides seguintes.
    \end{itemize}
\end{frame}

\begin{frame}{8º quadro: Repetindo a numeração do overlay (auto)}
     % '.' can be read as ‘repeat the overlay number of the last +’. So the two '+' overlay specifications create two slides, while the two lines using '.' in the specification ‘pick up’ the overlay number of the preceding '+'
    
    \begin{itemize}
        \item<+-> Este item aparece no primeiro e todos os slides seguintes.
        \item<.-> Este item também aparece no primeiro e todos os slides seguintes.
        \item<+-> Este item aparece no segundo e todos os slides seguintes.
        \item<.-> Este item também aparece no segundo e todos os slides seguintes.
    \end{itemize}
\end{frame}

\begin{frame}{9º quadro: Repetindo a numeração do overlay (auto, reduzido)}
    \begin{itemize}[<+->]
        \item Este item aparece no primeiro e todos os slides seguintes.
        \item<.-> Este item também aparece no primeiro e todos os slides seguintes.
        \item Este item aparece no segundo e todos os slides seguintes.
        \item<.-> Este item também aparece no segundo e todos os slides seguintes.
    \end{itemize}
\end{frame}

\begin{frame}{10º quadro: Offsets}
    % To allow even more flexibility, beamer has the concept of an ‘offset’: and adjustment to the number that is automatically inserted.

    \begin{itemize}
        \item<+(1)-> Este item aparece no segundo e todos os slides seguintes.
        \item<+(1)-> Este item aparece no terceiro e todos os slides seguintes.
        \item<+-> Este item também aparece no terceiro e todos os slides seguintes.
    \end{itemize}
\end{frame}

\begin{frame}{11º quadro: Offsets (exemplo realístico)}
    % A more realistic example for where an offset is useful is the case of revealing items ‘out of order’

    \begin{itemize}
        \item<+-> Este item aparece no primeiro e todos os slides seguintes.
        \item<+-> Este item aparece no segundo e todos os slides seguintes.
        \item<.(-1)-> Este item aparece no primeiro e todos os slides seguintes.
        \item<.-> Este item aparece no segundo e todos os slides seguintes.
    \end{itemize}
\end{frame}

\begin{frame}{12º quadro: Offsets (exemplo realístico equivalente)}
    % A more realistic example for where an offset is useful is the case of revealing items ‘out of order’

    \begin{itemize}
        \item<+-> Este item aparece no primeiro e todos os slides seguintes.
        \item<.(1)-> Este item aparece no segundo e todos os slides seguintes.
        \item<.-> Este item aparece no primeiro e todos os slides seguintes.
        \item<+-> Este item aparece no segundo e todos os slides seguintes.
    \end{itemize}
\end{frame}

\end{document}