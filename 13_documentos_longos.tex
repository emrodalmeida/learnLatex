% https://www.learnlatex.org/pt/lesson-13
% https://www.learnlatex.org/pt/more-13

\documentclass[brazilian, 12pt]{book}
\usepackage{babel}
\usepackage[T1]{fontenc}
\usepackage[parfill]{parskip}
\usepackage{imakeidx}      % para criar um índice remissivo
\usepackage{biblatex}
\addbibresource{biblatex-examples.bib}


\title{Um livro de exemplo}
\author{John Doe \and Joe Bloggs}


% útil para compilar apenas partes do documento, especialmente em documentos longos como livros ou dissertações.
\IfFileExists{\jobname.run.xml}
{
% Escolhemos quais capítulos serão de fato compilados usando \includeonly, que como você pode ver recebe uma lista separada por vírgula de nomes de arquivo. Ao usar o \includeonly, você pode encurtar o tempo total de execução do seu documento e produzir um PDF ‘seletivo’ para revisão. Adicionalmente, a principal vantagem de \includeonly é que o LaTeX vai usar todas as referências cruzadas dos arquivos .aux de outros arquivos incluídos.
\includeonly{
  front,
%  chap1,
  chap2,
%  append
  }
}
% não há comandos específicos, mas o comentário sugere compilar o documento inteiro para gerar arquivos auxiliares
{    
% Inicialmente compilar o documento inteiro
% para gerar todos os arquivos .aux
}


\begin{document}

% \frontmatter muda a numeração das páginas para algarismos romanos
\frontmatter

% \input é bom para partes de um longo arquivo que não são capítulos separados. No exemplo, nós o usamos para separar a capa e contracapa, mantendo o arquivo principal curto e claro.
\begin{center}
The front cover
\end{center}
\begin{center}
The front cover
\end{center}
\maketitle
\begin{center}
\large
For \ldots
\end{center}

\begin{center}
Copyright 2020 learnlatex.
\end{center}
\tableofcontents

% também usamos \input para as partes ‘não-capítulo’ no início do nosso ‘livro’: coisas como o prefácio. Novamente, isso é para manter o arquivo principal limpo.
\chapter{Preface}
The preface text. See \cite{doody}.

% =========================
\mainmatter

% \include é bom para capítulos, então o usamos para capítulos completos; ele sempre inicia uma nova página.
\chapter{Introduction}
The first chapter text.
\chapter{Something}
The second chapter text.

% \appendix muda a numeração dos capítulos para A, B etc., então o primeiro comando \chapter depois de \appendix imprime Appendix A.
\appendix
\chapter*{Appendix}
The first appendix text.

% ========================
\backmatter
\printbibliography
\newpage
\begin{center}
The back cover
\end{center}

\end{document}