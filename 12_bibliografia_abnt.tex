\documentclass{article}
\usepackage[T1]{fontenc}
\usepackage[utf8]{inputenc}
\usepackage[brazilian]{babel}
\usepackage[hidelinks]{hyperref}

\usepackage[alf, abnt-emphasize=bf, abnt-etal-list=0]{abntex2cite}
\citeoption{abnt-full-initials=yes, 
            abnt-missing-year=void,
            abnt-doi=link}
\citebrackets()

\begin{document}

Citação indireta de um artigo que descreve um algoritmo para melhoria de imagem \cite{janani_image_2015}. \\

Citação indireta do trabalho de \citeonline{janani_image_2015} que descreve um algoritmo para melhoria de imagem. \\

Uma citação direta conforme \citeonline{el-mashed_target_2012} seria "The proposed RDA-FrFT takes its advantage of the property of the FrFT to resolve chirp signals with high precision". \\

Outra citação direta: "The proposed RDA-FrFT takes its advantage of the property of the FrFT to resolve chirp signals with high precision" \cite{el-mashed_target_2012}. \\

Citação de uma página específica do datasheet do NE555 que contém a pinagem do CI \cite[p. 3]{datasheet_555}.

%\bibliographystyle{abnt-alf}
\bibliography{teste}

\end{document}