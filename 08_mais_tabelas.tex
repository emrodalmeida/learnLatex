% https://www.learnlatex.org/pt/more-08

\documentclass[brazilian, 12pt]{article}
\usepackage{babel}
\usepackage[T1]{fontenc}
\usepackage{array}   % adiciona mais funcionalidades às tabelas do LaTeX
\usepackage{booktabs}  % fornece quatro tipos diferentes de borda
\usepackage{siunitx}   % fornece o tipo de coluna S para alinhamento dos números
\usepackage{tabularx}  % para ajuste da largura da tabela
\usepackage{longtable}  % para tabelas multi-páginas
\usepackage{threeparttable}  % permite colocar notas em um bloco da mesma largura da tabela - https://texdoc.org/pkg/threeparttable

\setlength{\tabcolsep}{1cm}   % ajuste do espaço entre as colunas - https://texdoc.org/serve/siunitx/0

\begin{document}

% ========================
\section{Aplicando estilos}

Como > e < podem ser usados para colocar coisas antes e depois dos conteúdos das células de uma coluna, você pode usar esses símbolos para adicionar comandos que afetam a aparência de uma coluna.

\emph{multicolumn} pode ser usado para que aprimeira célula não seja afetada porque ela é o cabeçalho da tabela.\\

\begin{tabular}{>{\itshape}l<{} *{2}{l}}
    \toprule
    \multicolumn{1}{l}{Animal} & Comida & Tamanho \\
    \midrule
    cão    & carne  & médio   \\
    cavalo & capim  & grande  \\
    sapo   & moscas & pequeno \\
    \bottomrule
\end{tabular}

\vspace{10pt}

% ========================
\section{Manipulando espaços entre colunas}

O espaço é definido com o parâmetro \emph{tabcolsep}. Devido ao fato de cada coluna ser preenchida em ambos os lados, você obtém um \emph{tabcolsep} em cada lateral da tabela, mais 2\emph{tabcolsep} entre cada duas colunas – um de cada coluna. O ajuste é feito com \emph{setlength} no preâmbulo do documento.

Você pode trocar esse espaço por algo à sua escolha usando @. Esse símbolo vai remover o espaçamento entre duas colunas e inserir o argumento entre elas. \\

\begin{tabular}{l@{ : }l@{\hspace{2cm}}l}
    Animal & Comida & Tamanho \\
    cão    & carne  & médio   \\
    cavalo & capim  & grande  \\
    sapo   & moscas & pequeno \\
\end{tabular}

\vspace{10pt}

O símbolo ! adiciona o argumento no centro do espaço entre as colunas \\

\begin{tabular}{l!{:} l l}
    Animal & Comida & Tamanho \\
    cão    & carne  & médio   \\
    cavalo & capim  & grande  \\
    sapo   & moscas & pequeno \\
\end{tabular}

\vspace{10pt}

% ========================
\section{Personalizando as bordas de booktabs}

Todas as bordas do pacote booktabs e \emph{addlinespace} suportam um argumento opcional entre colchetes com o qual você pode especificar a espessura da borda. Além disso, o comprimento da redução do traço de \emph{cmidrule} pode ser modificado especificando um comprimento entre chaves depois de r ou l. \\

\begin{tabular}{@{} lll@{}}
    \toprule[2pt]
    Animal & Comida & Tamanho \\
    \midrule[1pt]
    cão    & carne  & médio   \\
    \cmidrule[0.5pt](r{1pt}l{1cm}){1-2}
    cavalo & capim  & grande  \\
    sapo   & moscas & pequeno \\ 
    \bottomrule[2pt]
\end{tabular}

\vspace{10pt}

% ========================
\section{Alinhamento de números em colunas}

O alinhamento de números em tabelas pode ser feito usando o tipo de coluna \emph{S}, do pacote \emph{siunitx}. Células não-numéricas devem ser “protegidas” colocando o seu conteúdo entre chaves. \\

\begin{tabular}{SS}
    \toprule[2pt]
    {Valores} & {Mais Valores} \\
    \midrule[1pt]
    1        &   2.3456 \\
    1.2      &   34.2345 \\
    -2.3     &   90.473 \\
    40       &   5642.5 \\
    5.3      &   1.2e3 \\
    0.2      &   1e4 \\
    \bottomrule[2pt]
\end{tabular}

\vspace{10pt}

% ========================
\section{Especificando a largura total da tabela}

É quase sempre preferível formatar a tabela para caber em uma largura conforme explicado abaixo do que redimensionar a tabela usando \emph{resizebox} e comandos similares que vão produzir tamanhos de fontes e espessuras de linha inconsistentes.

\subsection{tabular*}

O ambiente \emph{tabular*} leva um argumento adicional de largura, que especifica a largura total da tabela. Espaço esticável deve ser adicionado à tabela usando o comando \emph{extracolsep}. Esse espaço é adicionado entre todas as colunas a partir daquele ponto no preâmbulo. \emph{extracolsep} é quase sempre usado com \emph{fill}, um espaço especial que estica o quanto for necessário.

\begin{center}
\begin{tabular}{cc}
    \hline
    A & B \\
    C & D \\
    \hline
\end{tabular}
\end{center}

\vspace{10pt}

\begin{center}
\begin{tabular*} {0.6\textwidth} {@{\extracolsep{\fill}} cc @{}}
    \hline
    A & B \\
    C & D \\
    \hline
\end{tabular*}
\end{center}

\vspace{10pt}

\begin{center}
\begin{tabular*}{\textwidth}{@{\extracolsep{\fill}}cc@{}}
\hline
A & B\\
C & D\\
\hline
\end{tabular*}
\end{center}

\vspace{10pt}

\subsection{tabularx}

O ambiente \emph{tabularx}, do pacote de mesmo nome, tem uma sintaxe similar a \emph{tabular*} mas, ao invés de ajustar o espaço entre as colunas, ajusta a largura das colunas do tipo X. Isso é equivalente a uma coluna \emph{p\{...\}} com um valor automaticamente determinado para a largura.

\begin{center}
\begin{tabular}{lp{2cm}}
    \hline
    A & B B B B B B B B B B B B B B B B B B B B B B B B\\
    C & D D D D D D D\\
    \hline
\end{tabular}
\end{center}

\vspace{10pt}

\begin{center}
\begin{tabularx}{0.5\textwidth}{lX}
    \hline
    A & B B B B B B B B B B B B B B B B B B B B B B B B\\
    C & D D D D D D D\\
    \hline
\end{tabularx}
\end{center}

\vspace{10pt}

\begin{center}
\begin{tabularx}{\textwidth}{lX}
    \hline
    A & B B B B B B B B B B B B B B B B B B B B B B B B\\
    C & D D D D D D D\\
    \hline
\end{tabularx}
\end{center}

\vspace{10pt}

% ========================
\section{Tabelas multi-páginas}

Um ambiente tabular cria uma caixa inquebrável, então ela deve ser pequena o suficiente para caber em uma página, e é geralmente colocada em um ambiente float table. Vários pacotes têm variações com sintaxe similar que permitem quebras de página na tabela. Aqui mostramos o pacote \emph{longtable}: \\

\vspace{20pt}

\begin{longtable}{cc}
\multicolumn{2}{c}{Uma tabela longa} \\
Left Side & Right Side \\
\hline
\endhead
\hline
\endfoot
aa & bb\\  
Texto & b\\  
a & b\\  
a & b\\  
a & b\\  
a & b\\  
a & bbb\\  
a & b\\  
a & b\\  
a & b\\  
a & b\\  
a & b\\  
a & b\\  
a & b b b b b b\\  
a & b b b b b\\  
a & b b\\  
Texto mais largo & b\\
\end{longtable}

\vspace{10pt}

% ========================
\section{Notas em tabelas}

Usa o pacote \emph{threeparttable} \\

\begin{table}[h]
\begin{threeparttable}
    \caption{Um exemplo}
    \begin{tabular}{ll}
        \hline
        Texto & 42\tnote{1}\\
        Mais texto & 24\tnote{2}\\
        \hline
    \end{tabular}
    \begin{tablenotes}
        \item [1] A primeira nota.
        \item [2] A segunda nota.
    \end{tablenotes}
\end{threeparttable}
\end{table}

\vspace{10pt}

% ========================
\section{Novos tipos de colunas}

O pacote \emph{array} permite construções como \emph{>\{\textbackslash bfseries\}c} para denotar uma coluna centralizada com texto em negrito

\newcolumntype{B}{>{\bfseries}c}

\begin{center}
\begin{tabular}{Bc}
    \hline
    A & B \\
    C & D \\
    \hline
\end{tabular}
\end{center}

\vspace{10pt}

% ========================
\section{Truques Verticais}

Frequentemente, ao invés de fazer uma célula cobrir múltiplas linhas, é melhor fazer uma única linha em que algumas células são divididas verticalmente usando ambientes \emph{tabular} dentro dessa célula \\

\begin{tabular}{lcc}
  \toprule
  Teste & \begin{tabular}{@{}c@{}}A\\a\end{tabular} & \begin{tabular}{@{}c@{}}B\\b\end{tabular} \\
  \midrule
  Conteúdo & vai & aqui \\
  Conteúdo & vai & aqui \\
  Conteúdo & vai & aqui \\
  \bottomrule
\end{tabular}

\vspace{10pt}

 Você pode controlar o alinhamento vertical usando o argumento opcional do ambiente \emph{tabular}; ele suporta usar \textbf{t}, \textbf{c}, ou \textbf{b} para alinhar ao topo, ao centro, ou à base, respectivamente\\

\begin{tabular}{lcc}
  \toprule
  Teste & \begin{tabular}[b]{@{}c@{}}A\\a\end{tabular} & \begin{tabular}[t]{@{}c@{}}B\\b\end{tabular} \\
  \midrule
  Conteúdo & vai & aqui \\
  Conteúdo & vai & aqui \\
  Conteúdo & vai & aqui \\
  \bottomrule
\end{tabular}

\vspace{10pt}

% ========================
\section{Espaçamento entre linhas da tabela}

\begin{center}
\begin{tabular}{cc}
\hline
Quadrado & $x^2$\\
\hline
Cubo & $x^3$\\
\hline
\end{tabular}
\end{center}

\vspace{10pt}

O melhor é apenas aumentar a altura das linhas, sem aumentar sua profundidade abaixo da linha base \\

\begin{center}
\setlength\extrarowheight{20pt}  % \extrarowheight é parâmetro do pacote array
\begin{tabular}{cc}
\hline
Quadrado& $x^2$\\
\hline
Cubo& $x^3$\\
\hline
\end{tabular}
\end{center}



\end{document}