% https://www.learnlatex.org/pt/lesson-07

\documentclass[brazilian, twocolumn]{article}
\usepackage[T1]{fontenc}
\usepackage{graphicx}
\usepackage{lipsum}
\usepackage{babel}

\begin{document}
Essa imagem
\begin{center}
    \includegraphics[height=2cm]{example-image}
\end{center}
é um PDF importado

\begin{center}
    \includegraphics[width=0.7\textwidth]{example-image}
\end{center}
Mais texto.

Você também pode redimensionar imagens usando um fator de escala com \textbf{scale}, ou girá-las usando \textbf{angle}. Outra coisa que você pode querer fazer é cortar (com \textbf{clip}) ou aparar (com \textbf{trim}) uma imagem.
\begin{center}
    \includegraphics[clip, trim = 0 0 50 90]{example-image}
\end{center}

\lipsum[1-4]

Local de teste.

% Você pode usar quatro especificadores diferentes:
%     h ‘aqui’ se possível (here)
%     t topo da página (top)
%     b final da página (bottom)
%     p uma página dedicada para floats (float page)
\begin{figure}[ht]
    \centering % Dentro de um float (como o figure) você deve usar \centering se quiser centralizar o conteúdo; isso evita que o ambiente center adicione espaçamento vertical indesejado.
    \includegraphics[width=0.5\linewidth]{example-image-a.png}
    \caption{Uma imagem de exemplo}
\end{figure}

\lipsum[6-10]

\begin{center}
    \includegraphics[width=0.4\linewidth, angle=45]{example-image}
\end{center}

\end{document}