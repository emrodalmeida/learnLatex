% https://www.youtube.com/watch?v=naNYRFrCXV0&list=PLa_2246N48_rk8nzKg6VDykSEU6svwv4r&index=2

\documentclass[aspectratio=43]{beamer}
\usetheme{AnnArbor}
\setbeamercovered{transparent}      % Faz com que as partes ainda não mostradas do \pause e \uncover fiquem transparentes (não afeta o \visible)
\usepackage[utf8]{inputenc}

\title{Título}
\subtitle{Subtítulo}
\author[A.]{Autor}
\institute[Inst.]{Instituição}
\date{Data}
\logo{Imagem de Logo}

\begin{document}

\begin{frame}
	\titlepage
\end{frame}

\begin{frame}
	\frametitle{Título do quadro}
	\framesubtitle{Subtítulo do quadro}
	
	Primeiro item:
	\[ a^2x + bx +c = 0\]
	
	Segundo item: $y = ax + b$
	
	Terceiro item:
	$$ e^\pi + 1 = 0 $$
	
	Quarto item:
	$$ x^2 + y^2 = 0 $$
	
\end{frame}

\begin{frame}
	\frametitle{Título do quadro}
	\framesubtitle{item a item usando \textbackslash pause (reserva espaço para cada item)}
	
	Primeiro item:
	\[ a^2x + bx +c = 0\] \pause
	
	Segundo item: $y = ax + b$ \pause
	
	Terceiro item:
	$$ e^\pi + 1 = 0 $$ \pause
	
	Quarto item:
	$$ x^2 + y^2 = 0 $$
	
\end{frame}

\begin{frame}
	\frametitle{Título do quadro}
	\framesubtitle{item a item usando \textbackslash uncover (reserva espaço para cada item)}
	
	\uncover<1>{Primeiro item:
	\[ a^2x + bx +c = 0\]}
	
	% Colocar \setbeamercovered{transparent} no preâmbulo
	\uncover<2>{Segundo item: $y = ax + b$}
	
	\uncover<3>{Terceiro item:
	$$ e^\pi + 1 = 0 $$}
	
	\uncover<4>{Quarto item:
	$$ x^2 + y^2 = 0 $$}
	
\end{frame}

\begin{frame}
	\frametitle{Título do quadro}
	\framesubtitle{item a item usando \textbackslash visible (reserva espaço para cada item)}
	
	\visible<1>{Primeiro item:
		\[ a^2x + bx +c = 0\]}
	
	\visible<2>{Segundo item: $y = ax + b$}
	
	\visible<3>{Terceiro item:
		$$ e^\pi + 1 = 0 $$}
	
	\visible<4>{Quarto item:
		$$ x^2 + y^2 = 0 $$}
	
\end{frame}

\begin{frame}
	\frametitle{Título do quadro}
	\framesubtitle{item a item usando \textbackslash only (sempre centralizado verticalmente)}
	
	\only<1>{Primeiro item:
		\[ a^2x + bx +c = 0\]}
	
	\only<2>{Segundo item: $y = ax + b$}
	
	\only<3>{Terceiro item:
		$$ e^\pi + 1 = 0 $$}
	
	\only<4>{Quarto item:
		$$ x^2 + y^2 = 0 $$}
	
\end{frame}

\begin{frame}
	\frametitle{Título do quadro}
	\framesubtitle{item a item usando \textbackslash uncover e mantendo cada item no slide}
	
	\uncover<1->{Primeiro item:
		\[ a^2x + bx +c = 0\]}
	
	% Colocar \setbeamercovered{transparent} no preâmbulo
	\uncover<2->{Segundo item: $y = ax + b$}
	
	\uncover<3->{Terceiro item:
		$$ e^\pi + 1 = 0 $$}
	
	\uncover<4->{Quarto item:
		$$ x^2 + y^2 = 0 $$}
	
\end{frame}

\begin{frame}
	\frametitle{Título do quadro}
	\framesubtitle{item a item usando \textbackslash uncover e mantendo os itens em slides diferentes}
	
	\uncover<1-3>{Primeiro item:
		\[ a^2x + bx +c = 0\]}
	
	% Colocar \setbeamercovered{transparent} no preâmbulo
	\uncover<2->{Segundo item: $y = ax + b$}
	
	\uncover<1, 4>{Terceiro item:
		$$ e^\pi + 1 = 0 $$}
	
	\uncover<3-4>{Quarto item:
		$$ x^2 + y^2 = 0 $$}
	
\end{frame}

\begin{frame}
	\frametitle{Título do quadro}
	\framesubtitle{lista não numerada no ambiente itemize + ação em um slide específico}
	
	\begin{itemize}	
	\item<1- | alert@3> Primeiro item:
		\[ a^2x + bx +c = 0\]
	
	\item<2- | alert@2-4> Segundo item: $y = ax + b$
	
	\item<3- | alert@3> Terceiro item:
		$$ e^\pi + 1 = 0 $$
	
	\item<4- | alert@4> Quarto item:
		$$ x^2 + y^2 = 0 $$
	
	\end{itemize}
	
\end{frame}
	
\begin{frame}
	\frametitle{Título do quadro}
	\framesubtitle{comando automatizado no itemize para executar a ação sempre no slide atual}
	
	\begin{itemize}	[<+- | alert@+>]  % a ação tb pode ser uncover, visible, only
		\item Primeiro item:
		\[ a^2x + bx +c = 0\]
		
		\item Segundo item: $y = ax + b$
		
		\item Terceiro item:
		$$ e^\pi + 1 = 0 $$
		
		\item Quarto item:
		$$ x^2 + y^2 = 0 $$
		
	\end{itemize}
\end{frame}

\begin{frame}
	\frametitle{Título do quadro}
	\framesubtitle{sobreposição com blocos}
	
	\begin{block}<1>{Título 1}
		Bloco 1
	\end{block}
	
	\begin{block}<2>{Título 2}
		Bloco 2
	\end{block}
	
	\begin{block}<3>{Título 3}
		Bloco 3
	\end{block}
	
	\begin{block}<4>{Título 4}
		Bloco 4
	\end{block}
\end{frame}

\begin{frame}
	\frametitle{Título do quadro}
	\framesubtitle{sobreposição com figuras}
	
	\begin{figure}
		\centering
		\includegraphics<1>[scale=0.3]{figuras/divisorTensao_20241001a.png}
		\includegraphics<2>[scale=0.3]{figuras/divisorTensao_20241001b.png}
		\includegraphics<3>[scale=0.3]{figuras/pullUpDown.png}
		\includegraphics<4>[scale=0.3]{figuras/RLCserie_20241001.png}
	\end{figure}
	
\end{frame}

\end{document}