% https://www.learnlatex.org/pt/lesson-11

\documentclass[brazilian, 12pt]{article}
\usepackage[T1]{fontenc}
\usepackage{lipsum}
\usepackage[parfill]{parskip}  % suprime a identação dos parágrafos e coloca o espaço em branco entre eles

\begin{document}

% ============================================
\section{Espaçamento entre parágrafos}
Um estilo comum é não ter indentação para parágrafos, mas ter uma ‘linha em branco’ entre eles. Podemos obter esse resultado com o pacote \textbf{parskip}: \\[20pt]
\lipsum[1-3]


% ============================================
\section{Espaçamento explícito}
Algum texto \hspace{1cm} mais texto.

\vspace{10cm}

Ainda mais texto.


% ============================================
\section{Formatação explícita}
\subsection{Trechos curtos}
Vamos nos divertir com fontes: \textbf{negrito}, \textit{itálico}, \textrm{romano}, \textsf{sans serif}, \texttt{monoespaçado} and \textsc{versalete}.

\subsection{Texto corrido}
Texto normal.

{\itshape
Este texto é itálico.

Este também:  o efeito não se limita a um parágrafo, portanto é necessário colocar o texto em um grupo.
}

Ambientes no LaTeX formam grupos, assim como células de tabelas, ou podemos usar \{...\} para criar um grupo explícito.

\subsection{Tamanho da fonte}
É obrigatório terminar um parágrafo antes de mudar o tamanho da fonte de volta. Isso pode ser feito adicionando um \textbf{\textbackslash par}, que equivale a uma linha em branco.

\begin{center}
{\itshape \huge Algum texto\par}
Texto normal\\
{\bfseries \tiny Texto muito menor\par}
\end{center}


% ============================================
\section{Suprimindo a indentação de um parágrafo}
Se você quer omitir a indentação de um único parágrafo, você pode usar \textbf{\textbackslash noindent}. Esse comando deve ser usado raramente; na maioria das vezes você deve deixar o LaTeX cuidar disso automaticamente.

Um pequeno parágrafo, que nós alongamos para ter certeza que você consiga ver o efeito aqui!

Um pequeno parágrafo, que nós alongamos para ter certeza que você consiga ver o efeito aqui!

\noindent Um pequeno parágrafo, que nós alongamos para ter certeza que você consiga ver o efeito aqui!

\end{document}